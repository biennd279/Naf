% \documentclass[12pt, a4paper]{article}
\documentclass[12pt,a4paper]{report}
\usepackage[utf8]{vietnam}
\usepackage[T5]{fontenc}

\usepackage{amsmath}
\usepackage{amsfonts}
\usepackage{amssymb}
\usepackage{makeidx}
\usepackage{imakeidx}
\usepackage{graphicx}
\usepackage{graphics}
\usepackage{placeins}
\usepackage[unicode, bookmarksopenlevel=4]{hyperref}
\usepackage{makeidx}
\usepackage[style=alphabetic]{biblatex}
\usepackage{multicol}
\usepackage{subfiles}
\usepackage{hyperref}
\usepackage{enumitem}
\usepackage{float}
\usepackage[table,xcdraw]{xcolor}
\usepackage{tabularx}
\usepackage{wrapfig}
\usepackage{caption}
\usepackage{subcaption}
\usepackage{placeins}
\usepackage{array}
\usepackage{longtable}
\usepackage{multirow}
\usepackage{tikz}
\usepackage{pgfplots}
\usetikzlibrary{calc}

\let\orgautoref\autoref
\def\code#1{\texttt{#1}}

\setcounter{secnumdepth}{4}
\setcounter{tocdepth}{2}

\newcommand{\iindex}[1]{\textit{#1}\index{#1}}

% Create file reference.bib to add
\addbibresource{./reference.bib}

\graphicspath{ {./images/} {./../images}}
\DeclareGraphicsExtensions{.png,.eps,.svg}
\setlist[description]{leftmargin=\parindent,labelindent=\parindent}

\title{Phát triển một công cụ hỗ trợ phát hiện và khai thác tự động các lỗ hổng Web}

\pagenumbering{roman}
\begin{document}

\subfile{cover.tex}
\clearpage{}

\chapter*{Tóm tắt}

Zed Attack Proxy (ZAP) là một công cụ mã nguồn mở giúp rà soát lỗ hổng tự động theo tập luật được phát triển liên tục bởi cộng đồng. Tuy nhiên, ZAP lại thiếu đi các công cụ xác minh và khai thác các lỗ hổng. Công cụ Naf được phát triển sẽ giúp bổ sung cho ZAP về việc xác minh và khai thác lỗ hổng phía máy chủ Web tự động.

Hiện tại với công cụ, đã có thể xác minh được các lỗng hổng mức nguy hiểm như SQL Injection, Command Injection, Remengineote File Inclusion, Server Side Template Injection,... với độ chính xác cao, đảm bảo được tỉ lệ dương tính giả thấp, từ đó lược bỏ được phần lớn công đoạn xác minh lỗ hổng đối với các lỗ hổng nguy hiểm này. So với các công cụ thương mại trên thị trường như Invicti (trước đây là Netsparker) thì công đã bổ sung cho ZAP tính năng tương đương, với độ tin tưởng cao.

Công cụ có sử dụng các bộ công cụ mã nguồn mở phổ biến như SQLMap, Commix, TplMap,... làm engine để thực hiện quá trình xác minh, cung cấp thêm các công cụ khai thác tự động. Do đều là các công cụ mã nguồn mở được bảo trì liên tục nên có khả năng khai thác được miền rất rộng các lỗ hổng với kỹ thuật được cập nhật

\chapter*{Lời cảm ơn}

Lời đầu tiên cho phép em được gửi lời cảm ơn tới Khoa Công Nghệ Thông Tin – Trường Đại học Công Nghệ ĐHQG Hà Nội đã tạo điều kiện thuận lợi cho em được học tập, nghiên cứu và thực hiện đề tài tốt nghiệp này.

Em cũng xin được bày tỏ lòng biết ơn sâu sắc tới thầy Nguyễn Đại Thọ đã tận tình hướng dẫn, đóng góp những ý kiến giúp em hoàn thành khóa luận tốt nghiệp một cách tốt nhất.

Em cũng vô cùng biết ơn các thầy cô trong trường tận tình giảng dạy, truyền thụ cho em những kiến thức và kỹ năng quan trọng làm hành trang vững chắc trên con đường học vấn của bản thân.

Chúc mọi người luôn luôn mạnh khoẻ và gặt hái được nhiều thành công trong cuộc sống.

\chapter*{Lời cam đoan}

Em xin cam đoan rằng khóa luận tốt nghiệp này không sao chép từ bất kỳ ai, tổ chức nào khác. Toàn bộ nội dung được trình bày trong tài liệu đều là cá nhân em qua quá trình học tập, tìm hiểu và nghiên cứu mà hoàn thiện. Mọi tài liệu tham khảo đều được ghi chép lại và trích dẫn hợp pháp. Nếu lời cam đoan là sai sự thật thì em xin chịu mọi trách nhiệm và hình thức kỷ luật theo quy định từ phía nhà trường.

\tableofcontents{}
\clearpage{}

\listoffigures{}

\listoftables{}

\chapter{Mở đầu}
\pagenumbering{arabic}

\section{Đặt vấn đề}
\subfile{./sections/section_1.tex}


\nocite{*}
\printbibliography[heading=bibintoc, title=Tài liệu tham khảo]

% \chapter*{Từ điển chú giải}

\end{document}

