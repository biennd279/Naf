% \documentclass[12pt, a4paper]{article}
\documentclass[12pt,a4paper]{report}
\usepackage[utf8]{vietnam}
\usepackage[T5]{fontenc}

\usepackage{amsmath}
\usepackage{amsfonts}
\usepackage{amssymb}
\usepackage{makeidx}
\usepackage{imakeidx}
\usepackage{graphicx}
\usepackage{graphics}
\usepackage{placeins}
\usepackage[unicode, bookmarksopenlevel=4]{hyperref}
\usepackage{makeidx}
\usepackage[style=alphabetic]{biblatex}
\usepackage{multicol}
\usepackage{subfiles}
\usepackage{hyperref}
\usepackage{enumitem}
\usepackage{float}
\usepackage[table,xcdraw]{xcolor}
\usepackage{tabularx}
\usepackage{wrapfig}
\usepackage{caption}
\usepackage{subcaption}
\usepackage{placeins}
\usepackage{array}
\usepackage{longtable}
\usepackage{multirow}
\usepackage{tikz}
\usepackage{pgfplots}
\usepackage{listings,xcolor}

\definecolor{dkgreen}{rgb}{0,.6,0}
\definecolor{dkblue}{rgb}{0,0,.6}
\definecolor{dkyellow}{cmyk}{0,0,.8,.3}

\usetikzlibrary{calc}

\let\orgautoref\autoref
\def\code#1{\texttt{#1}}

\setcounter{secnumdepth}{4}
\setcounter{tocdepth}{2}

\newcommand{\iindex}[1]{\textit{#1}\index{#1}}
\renewcommand{\lstlistingname}{Đoạn mã}% Listing -> Algorithm
\renewcommand{\lstlistlistingname}{Danh sách các \lstlistingname}

% Create file reference.bib to add
\addbibresource{./reference.bib}

\graphicspath{ {./images/} {./../images}}
\DeclareGraphicsExtensions{.png,.eps,.svg}
\setlist[description]{leftmargin=\parindent,labelindent=\parindent}

\title{Phát triển một công cụ hỗ trợ phát hiện và khai thác tự động các lỗ hổng Web}

\lstset{
  language        = php,
  basicstyle      = \small\ttfamily,
  keywordstyle    = \color{dkblue},
  stringstyle     = \color{red},
  identifierstyle = \color{dkgreen},
  commentstyle    = \color{gray},
  emph            =[1]{php},
  emphstyle       =[1]\color{black},
  emph            =[2]{if,and,or,else},
  emphstyle       =[2]\color{dkyellow},
  breakatwhitespace =   false,
  breaklines    = true
  }

\pagenumbering{roman}
\begin{document}

\subfile{cover.tex}
\clearpage{}

\chapter*{Tóm tắt}

Zed Attack Proxy (ZAP) là một công cụ mã nguồn mở giúp rà soát lỗ hổng tự động theo tập luật
được phát triển liên tục bởi cộng đồng. Tuy nhiên, ZAP lại thiếu đi các công cụ xác minh và
khai thác các lỗ hổng. Công cụ Nextgen Automation Framework (NAF) được phát triển trong
khuôn khổ của khóa luận này sẽ giúp bổ sung cho ZAP về việc xác minh và khai thác lỗ hổng
phía máy chủ Web tự động.

Các công cụ khai thác sẵn có như Invicti, Acunetix,\ldots cho phép xác minh một cách tự động các
lỗ hổng Web nghiêm trọng như SQL Injection, Command Injection, Remote File Inclusion,
Server Side Template Injection,\ldots là thực sự tồn tại giúp công tác kiểm thử an ninh được thực
hiện một cách hiệu quả hơn, giảm thiểu các thao tác thủ công phức tạp tiêu tốn nhiều thời gian.

Công cụ NAF của chúng tôi tích hợp vào ZAP các bộ công cụ mã nguồn mở phổ biến như SQLMap,
Commix, Tplmap,\ldots biến ZAP trở thanh một công cụ vừa có chức năng dò quét vừa có chức năng
xác minh tự động các lỗ hổng an ninh Web và cung cấp thêm các công cụ khai thác. Do đều là các
công cụ mã nguồn mở được bảo trì liên tục nên có khả năng khai thác được miền rất rộng các
lỗ hổng với kỹ thuật được cập nhật. Do còn phụ thuộc vào các tập luật được phát triển bởi cộng
đồng và chưa có khả năng phát hiện được một số lỗ hổng khi rà quét nhưng NAF lại có khả năng xác
minh và khai thác một số lỗ hổng nhiều hơn so với các công cụ thương mại như Invicti điển hình
là lỗ hổng Server Side Template Injection.

\chapter*{Lời cảm ơn}

Lời đầu tiên cho phép em được gửi lời cảm ơn tới Khoa Công Nghệ Thông Tin - Trường Đại học Công
Nghệ ĐHQG Hà Nội đã tạo điều kiện thuận lợi cho em được học tập, nghiên cứu và thực hiện
đề tài tốt nghiệp này.

Em cũng xin được bày tỏ lòng biết ơn sâu sắc tới thầy Nguyễn Đại Thọ đã tận tình hướng dẫn,
đóng góp những ý kiến giúp em hoàn thành khóa luận tốt nghiệp một cách tốt nhất.

Em cũng vô cùng biết ơn các thầy cô trong trường tận tình giảng dạy, truyền thụ cho
em những kiến thức và kỹ năng quan trọng làm hành trang vững chắc trên con đường học vấn
của bản thân.

Chúc mọi người luôn luôn mạnh khoẻ và gặt hái được nhiều thành công trong cuộc sống.

\chapter*{Lời cam đoan}

Em xin cam đoan rằng khóa luận tốt nghiệp này không sao chép từ bất kỳ ai,
tổ chức nào khác. Toàn bộ nội dung được trình bày trong tài liệu đều là cá nhân em qua quá
trình học tập, tìm hiểu và nghiên cứu mà hoàn thiện. Mọi tài liệu tham khảo đều được ghi chép
lại và trích dẫn hợp pháp. Nếu lời cam đoan là sai sự thật thì em xin chịu mọi trách nhiệm và
hình thức kỷ luật theo quy định từ phía nhà trường.

\tableofcontents{}
\clearpage{}

\listoffigures{}

\listoftables{}

\lstlistoflistings

\chapter{Mở đầu}
\pagenumbering{arabic}

\subfile{./sections/section_1.tex}

\chapter{Kiểm thử an ninh ứng dụng Web}

\section{Lỗ hổng ứng dụng Web}
\subfile{./sections/section_2.tex}

\chapter{Phát triển NAF - Một công cụ rà quét và xác minh tự động các lỗ hổng Web}
\subfile{./sections/section_3.tex}

\chapter{Thử nghiệm và đánh giá}
\subfile{./sections/section_4.tex}

\chapter{Kết luận}
\subfile{./sections/section_5.tex}

\appendix
\chapter{Phụ lục mã nguồn ứng dụng chứa lỗ hổng}
\subfile{./sections/section_6.tex}



\nocite{*}
\printbibliography[heading=bibintoc, title=Tài liệu tham khảo]

% \chapter*{Từ điển chú giải}

\end{document}

