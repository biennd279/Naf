% !TeX root = ../main.tex
\documentclass[./../main.tex]{subfiles}

\begin{document}
\subsection{Tổng quan ứng dụng}

Dù rằng ZAP đã cung cấp một mô hình tương đối hoàn chỉnh để thực hiện được
việc kiểm thử an toàn thông tin, nhưng phần lớn các ZAP Addon lại rất
phụ thuộc vào việc được cung cấp thông tin về Website để mô hình thành
cây Site Node từ cơ chế “middle-in-the-middle-proxy" tức là người kiểm
thử an ninh cần sử dụng trực tiếp trên trình duyệt Web một thời lượng lớn
dù rằng các ZAP Addon có thể thay thế một phần việc này.

Một vấn đề khác là do các ZAP Addon được thiết kế để chạy được độc lập
nên sự tương tác giữa các ZAP Addon chưa cao dù rằng chỉ với một số Addon
trong ZAP Core đã có thể hoàn thành lượng lớn công việc đánh giá lỗ hổng.

ZAP cũng chỉ xây dựng được cơ chế cảnh báo từ việc thực hiện đánh giá
lỗ hổng thông qua các luật, điều này là không đủ để sinh ra được báo
cáo hoàn chỉnh mục đích đưa ra được Proof-of-Concept cho lỗ hổng, cũng
như cách xử lý cho nhà phát triển.

Ngoài ra, các ZAP Addon cũng không thể cung cấp được các công cụ quét
lỗ hổng chuyên sâu hoặc lỗ hổng mức hệ thống và thiếu vắng đi các công
cụ khai thác tự động để đưa ra được Proof-of-Exploit cho lỗ hổng.

Do đó, công cụ AVA được sinh ra để thống nhất việc tương tác giữa các
Addon để nâng mức tự động lên mức độ chấp nhận được và bổ sung thêm
các chi tiết, tính năng thiếu đã nêu ở trên để phù hợp đây là một công
cụ tự động xác minh và khai thác tự động.

\subsection{Các component}
Các thành phần sẽ được mô hình hóa thành các Component. Mỗi Component
sẽ phụ trách một Logic riêng theo từng phần của công cụ. Các Component
này phụ thuộc nhau dạng cây như hình \ref{fig:component} và được tổ chức
mã nguồn như hình \ref{fig:package}.

\begin{figure}[h!]
	\includegraphics[width=\linewidth]{./images/component.png}
	\caption{Biểu đồ phụ thuộc của các Component}
	\label{fig:component}
\end{figure}

Đối với ngôn ngữ triển khai, em lựa chọn Kotlin là một ngôn ngữ JVM
có thể tương tác được với Java.

\begin{figure}[h!]
	\includegraphics[width=\linewidth]{./images/package.png}
	\caption{Tổ chức component trong mã nguồn}
	\label{fig:package}
\end{figure}

Về triển khai giao diện người dùng và tổ chức mã nguồn, em lựa chọn
Jetpack Compose và Decompose làm các thư viện chính và tổ chức Component
thay cho Swing - một thư viện khá cũ và khó tổ chức khi đưa cả logic và
giao diện người vào chung một mã nguồn.

\subsection{Kiến trúc kiểm thử tự động theo đường ống}
Như đã nêu vấn đề ở trên, ZAP sử dụng các Addon như một phần nên việc
thích hợp các công cụ xác minh tự động khá khó. Do đó, em xây dựng một
đường ống (Pipeline) để tương tác được với ZAP Core, sử dụng được với các
ZAP Addons khác, đồng thời có thể thêm các chi tiết khác vào quá trình kiểm
thử tự động.

Mỗi đoạn ống sẽ thuộc về một pha xác định của kiểm thử, pha có độ ưu tiên
cao được thực hiện trước để cung cấp thông tin cho các pha sau.

Các pha bao gồm: Khởi tạo, Kiểm thử mờ, Cào dữ liệu, Rà quét lỗ hổng.

Các kiến trúc này xoay quanh các khai niệm sau:
\begin{itemize}
	\item Ngữ cảnh quét (Context) là thành phần chứa thông tin của các cuộc
	      tìm kiếm lỗ hổng tự động, các đoạn của đường ống sẽ sử dụng thông tin này để thực hiện các hoạt động của mình một cách động lập trên toàn đường ống. Ngữ cảnh quét ánh xạ tới các thành phần ZAP Core.
	\item Đường ống (Pipeline) là quá trình liên tục chứa các đoạn, tại đây
	      diễn ra việc tìm lỗ hổng tự động và xác minh tự động.
	\item Các đoạn của đường ống (Stage) là các đơn vị chạy các việc để thực
	      hiện quá trình kiểm thử tự động.
\end{itemize}

\subsubsection{Khởi tạo}

Đối với các đoạn của đường ống là đoạn khởi tạo ngữ cảnh thì cần có tương
tác trực tiếp với cả ZAP Core và cả đường ống.

Các tương tác này được xử lý như sau
\begin{itemize}
	\item Thiết lập ngữ cảnh riêng cho việc kiểm thử: Mỗi lần kiểm
	      thử tự động nên có một ngữ cảnh riêng ứng với đó là một ngữ cảnh
	      của ZAP riêng chứa các thông tin về thông tin cho việc sử dụng
	      Proxy, Crawler, Scanner,...
	\item Thiết lập mục tiêu (Target) cho kiểm thử: Thực hiện truy cập
	      vào truy vấn tới URL mục tiêu, cung cấp cho hệ thống thêm thông tin
	      về Site Node gốc, đánh dấu điểm bắt đầu của Sitemap.
	\item Thông tin về các ca kiểm thử được sử dụng. Thông tin được
	      đánh dấu dưới dạng Chính sách (Policy) \ref{fig:policy}
	      \begin{itemize}
		      \item Về công nghệ sử dụng: Được cung cấp qua tập công
		            nghệ sử dụng (Tech Set) trong ZAP Context, Scanner sẽ
		            sử dụng thông tin này để lựa chọn phù hợp để kiểm thử,
		            đối với trường hợp thông thường tập công nghệ được
		            đánh dấu toàn bộ sẽ khiến lượng ca kiểm thử trở lên rất lớn.
		      \item Về ngưỡng cảnh báo (Threshold): Các ca kiểm thử đều có ngưỡng
		            cảnh báo riêng dựa trên hành vi của hệ thống đối với ca
		            kiểm thử, mức độ càng thấp tăng lượng âm tính giả càng
		            lớn ngược lại làm tăng mức dương tính giả.
		      \item Mức độ cảnh báo (Strength): Thiết lập để phù hợp với các mục
		            tiêu kiểm thử, mức độ này sẽ được trả về hệ thống thông
		            qua cơ chế Thông báo (Alert)
	      \end{itemize}
	      \begin{figure}[h!]
		      \includegraphics[width=\linewidth]{./images/policy.png}
		      \caption{ Điều chỉnh các các chính sách }
		      \label{fig:policy}
	      \end{figure}
	\item Thông tin về xác thực: Cần được thực hiện một lần đăng nhập
	      để lấy thông tin để cung cấp đầy đủ thông tin cho ZAP Context \ref{fig:authen}.
	      ZAP Context sẽ “hướng dẫn” Crawler, Scanner thực hiện đăng nhập khi
	      bị đăng xuất. Các thông tin cần được cung cấp bao gồm:
	      \begin{itemize}
		      \item Trang đăng nhập
		      \item Thông tin về yêu cầu đăng nhập
		      \item Thông tin ủy nhiệm (Credential)
		      \item Biểu hiện khi đã đăng nhập/đăng xuất.
	      \end{itemize}
	      \begin{figure}[h!]
		      \includegraphics[width=\linewidth]{./images/authen.png}
		      \caption{ Đưa ra thông tin xác thực }
		      \label{fig:authen}
	      \end{figure}

\end{itemize}
Với các thông tin trên nên được thực hiện tự động và ẩn đi phần logic
phức tạp trên, người dùng chỉ nên được tương tác thông qua một trình
Wizard.

Các thông tin này được em triển khai bằng cách giữ lại nhưng một
trạng thái có thể lưu lại, chỉ được thực hiện khi bắt đầu quá trình
kiểm thử tự động. Khi kiểm thử tự động sẽ dựa vào các thông tin này
để sinh ra các thiết lập để đưa vào ZAP Core một cách tự động.

\subsubsection{Kiểm thừ mờ}
Một việc cần thiết là để ZAP có thể hoạt động hiệu quả cung cấp thêm
thông tin về cấu trúc của Website cho ZAP. Một trong những phương pháp
phổ biến cho việc này là vét cạn các điểm cuối thường có của một Website
theo từ điển (thường được sử dụng với tên Fuzzing). Kỹ thuật này hiệu
quả với việc kiểm thử hộp đen và hộp xám khi không có hoặc có rất ít
các thông tin về hệ thống. Ngoài ra, nó cũng giúp phát hiện ra có lỗ
hổng do việc thiết lập cấu hình không an toàn.

Cộng đồng phát triển ZAP đã phát triển một Addons là Forced Browsing trên
cơ sở sử dụng thư viện DirBuster có hiệu quả khá cao trong việc kiểm thử mờ
cũng như cung cấp một vài danh sách các từ phổ biến cho việc kiểm thử mờ.
Đoạn của đường ống này sẽ sử dụng Addon trên và cung cấp danh sách các từ
(người dùng lựa chọn) để thực hiện việc kiểm thử này. Việc này cũng được
thiết đặt tự động trong trình Wizard thay vì phải thiết lập thủ công \ref{fig:fuzz}.

Đây là một trong đoạn đầu tiên của đường ống, nên chưa có nhiều thông tin
về ngữ cảnh được cung cấp, nên chỉ cần dựa trên danh sách các từ được người
dùng  lựa chọn và các tùy chọn để thực hiện kiểm thử mờ và cố gắng thu thập
lại lượng endpoint phù hợp để sinh ra lượng Site Node lớn nhất để có
thông tin về Website.

\begin{figure}[h!]
	\includegraphics[width=\linewidth]{./images/fuzz.png}
	\caption{Tùy chọn danh sách các từ để kiểm thử mờ}
	\label{fig:fuzz}
\end{figure}

\subsubsection{Cào dữ liệu}

Một cách nữa để cung cấp thông tin về cấu trúc của Website là
cào (Crawl) dữ liệu trên toàn bộ Website để lấy thông tin.

Đội ngũ phát triển ZAP cũng đã phát triển hai Addon là Spider và
Ajax Spider. Spider được sử dụng để cào dữ liệu, thông tin lấy từ cây
DOM của Website trong các phản hồi của các thông báo HTTP.

Đối với một số Website sẽ sử dụng thêm Ajax để lấy thông tin từ máy
chủ, các trình cào dữ liệu tĩnh như Spider sẽ không phát hiện được các yêu cầu này.
Đội ngũ phát triển ZAP cũng đã phát triển thêm Ajax Spider, công cụ xây
dựng dựa trên Selenium để thực hiện việc cào dữ liệu trực tiếp bằng
Selenium đấy được thông tin về các yêu cầu thực hiện thông qua Ajax
bằng cách thực thi Javascript như một trình duyệt thực.

Hai Addon này cần sử dụng trực tiếp các thông tin từ ngữ cảnh của ZAP Core.
Để tự động được các Addon này cần đồng thời xử lý thông tin ở ZAP Context
trước khi gửi thông tin cho Addon và thực thiện gọi các Addon thực hiện
theo trình tự của của các pha.

Các đoạn của đường ống này có mục tiêu là tối ưu hóa việc cào dữ liệu từ
DOM và các yêu cầu Ajax. Để chúng hoạt động tốt nhất cần cung cấp
ZAP Context chứa các thông tin về phạm vi và thông tin đăng nhập để quá
trình cào dữ liệu không bị hủy bỏ do vấn đề xác thực.

\subsubsection{Rà quét lỗ hổng}

Đây là pha chính trong việc kiểm thử an ninh, tại đây các luật (Rule)
của cộng đồng đóng góp sẽ được sử dụng để kiểm tra các hành vi của hệ
thống để suy diễn ra các lỗ hổng của hệ thống. Cộng đồng phát triển ZAP
đã phát triển hai Addons là Passive Scan và Active Scan. Tuy nhiên,
Addon Passive Scan lại chỉ phù hợp khi thực hiện kiểm thử thủ công hoặc bán
tự động do chỉ rà quét các thông tin bằng cách rà quét thụ động. Addon Active Scan
là công cụ hiệu quả cho việc chạy các luật được đưa ra của chính sách trong
ngữ cảnh của ZAP. Chính công cụ này sẽ thực hiện việc thử các luật, loại bỏ
bớt các luật dựa trên các tập công nghệ được đưa vào, lựa chọn các ca kiểm
thử phù hợp với ngưỡng quét và đưa ra cảnh báo dựa trên mức độ cảnh bảo.

Đoạn của đường ống hoạt động này dựa trên chính sách đã được cung cấp để đưa ra
cảnh báo. Quá trình này cũng cần được cung cấp ZAP Context để có thể thực
hiện ca kiểm thử mà không hủy bỏ do vấn đề xác thực. Đồng thời cũng nên đưa
vào các vùng nguy hiểm trong phạm để tránh làm lỗ hệ thống (ví dụ các
thiết lập tài khoản, hệ thống).

Việc cài đặt bước này cần đem được các thông tin từ pha trước kết hợp
với thông tin người dùng cung cấp để thực hiện. Các Addons tương đối độc
lập nên thông tin đầu ra của từng Addons là khác nhau và không ổn định,
các thông tin đầu ra được thu thập, chuẩn hóa và đưa lại vào trong AVA
để trực quan hóa thông tin \ref{fig:scan_result}.

\begin{figure}[h!]
	\includegraphics[width=\linewidth]{./images/scan_result.png}
	\caption{Kết quả crawl}
	\label{fig:scan_result}
\end{figure}

\subsection{Thích hợp công cụ xác minh tự động}

Các công cụ xác minh tự động làm nhiệm vụ xác minh lại các cảnh bảo,
nếu được xác minh, một Vấn đề (Issue) mới được tạo ra và được thêm vào cơ sở dữ
liệu.

Để xác định được cảnh báo nào thuộc về lỗ hổng nào cần sử dụng đến
giá trị định danh của Common Weakness Enumeration (CWE) để xác định
tương đối lỗ hổng nào thuộc về loại nào.

Cần xem quét cách mà các luật được sử dụng để đưa ra được cách xác minh phù
hợp với từ loại lỗ hổng. Trong phạm vi của công cụ, các lỗ hổng đã được
xem xét thông qua mã nguồn của các luật được sử dụng trong rà quét chủ động
bao gồm:
\begin{description}
	\item[SQL Injection] Luật thiết lập ZAP chủ động gửi các trọng tải mang
	      gây lỗi cho các truy vấn SQL như ', \textbackslash, ;, (, ), NULL, \ldots
	      với các trọng tải khác phù hợp với các kĩ thuật như Error-based, Boolean-Based,
	      Union-Based,\ldots Và sẽ lấy các thông báo phản hồi bất thường như
	      các lỗi, thông tin trả về như một Proof-Of-Concept để thông báo tới
	      người dùng.
	\item[Command Injection] Luật cũng đưa vào các dấu ngắt lệnh như ', ;, | \ldots
	      để cố gắng chạy thêm các lệnh phù hợp với từ hệ điều hành. Luật cũng
	      sử dụng các hành vi của máy chủ Web để xem xét đưa ra Proof-Of-Concept
	      phù hợp với các kĩ thuật tấn công như Feedback-based hay Blind Time-based.
	\item[Code Injection] Luật chỉ tập trung vào việc xác minh Remote Code Evaluation.
	      bằng cách chèn thêm các trọng tải chứa các đoạn mã thực thi đưa
	      vào phản hồi các đoạn đánh dấu đặc biệt và kiểm tra xem phản hồi
	      có cả đoạn mã này hay không.
	\item[Path Traversal] Luật đưa gửi các trọng tải liên quan đến việc truy
	      cập các tập tin, đặc biệt của hệ điều hành như /etc/passwd, hay system.ini,
	      khi các thông tin của tập tin được trả về thì có thể xác định ngay xem
	      có bị lỗ hổng hay không.
	\item[Remote File Include] Luật đưa vào một đường dẫn điển hình tới
	      chứa trọng tải tới một trang Web và cố xác định xem trang Web đó có được
	      thêm vào hay không. Tuy nhiên luật này sẽ có trường hợp dương tính giả
	      với lỗ hổng Server-side Request Forgery do không xác định xem việc
	      thêm vào có được thực thi hay không.

\end{description}


Qua các triển khai của luật trên, các lỗ hổng cần được xác minh cần được
phân phối cho các công cụ và được triển khai như sau:
\begin{description}
	\item[SQL Injection] SQLMap
	\item [Command Injection/Code Injection] Commix
	\item [Code Injection] Tplmap
	\item [Remote File Inclusion] RFIExploiter
	\item [Local File Inclusion/Path Traversal] LFIExploiter
\end{description}

\subsubsection{SQLMap}
Công cụ SQLMap, đi kèm với một API Server nên việc giao tiếp có thể thiết
lập qua công API Server này. API Server sử dụng là REST-API sử dụng JSON
là format để truyền tải thông tin. Tuy nhiên dữ liệu trả về không là có
định dạng chuẩn do SQLMap được xây dựng bằng Python là một ngôn ngữ kiểu
động. Do đó, cần xây dựng thêm một số Parser riêng để có thế đưa dữ liệu
trả về đúng kiểu \ref{fig:sqlmap}.

Lúc này các nhiệm vụ khai thác được mô hình hóa thành các Job, API Server
sẽ xử lý các Job trong Job Pool và trả về dữ liệu theo các yêu cầu của công
cụ gửi lên. Các yêu cầu có thể là xác minh lỗ hổng, lấy các thông tin của
cơ sở dữ liệu.


\begin{figure}[h!]
	\includegraphics[width=\linewidth]{./images/SQLMap.png}
	\caption{Biểu đồ tương tác giữa SQLMap và công cụ.}
	\label{fig:sqlmap}
\end{figure}

Đối với pha xác minh, SQLMap chỉ cần xác minh là lỗ hổng có tồn tại
hay không dựa trên thông tin cảnh báo đưa ra.
\subsubsection{Commix và Tplmpa}
Khác với SQLMap, Commix và Tplmap không có một API Server, các tương
tác cần được đưa vào và ra thông qua đầu vào chuẩn (STDIN) và đầu ra
chuẩn(STDOUT) \ref{fig:commix_tplmap}.

Đối với Docker Engine cũng cần có API giao tiếp gián tiếp để có thể thực
thi và lấy được đầu vào chuẩn và đầu ra chuẩn của Container.


\begin{figure}[h!]
	\includegraphics[width=\linewidth]{./images/DockerIO.png}
	\caption{Biểu đồ tương tác công cụ với Commix và Tplmap.}
	\label{fig:commix_tplmap}
\end{figure}

Đối với việc xác minh, Commix và Tplmap sẽ cố gắng thực thi một lệnh
và kiểm tra lệnh có được thực thi hay không bằng kiểm tra đầu ra chuẩn
để xác minh.

\subsubsection{Remote File Inclusion}
Lỗ hổng này là đặc trưng của PHP nên sẽ chỉ được kiểm tra với PHP. Đưa
đường dẫn tới một pha đầu vào chứa một đoạn mã thực thi của PHP, nếu
được thực thi thì được xác minh là đúng. Với lỗ hổng này không có các
dự án mã nguồn mở để khai thác nên em cần tự triển khai một trình khai
thác riêng.
\subsubsection{Local File Inclusion/Path Traversal}
Lỗ hổng này sẽ không cần xác minh lại do độ, nhưng vẫn được xây dựng để
phát triển công cụ khai thác. Lỗ hổng này cũng không có dự án mã nguồn
mở để khai thác nên em cũng tự triển khai một trình khai thác riêng.

\subsection{Trực quan hóa thông tin}
Dù là các ZAP Addons cung cấp khá đầy đủ các thông tin về thông tin của
chính Addons đó cho người dùng nhưng chính sự độc lập của các Addons lại
khiến thông tin của toàn bộ ứng dụng ZAP nói chung lại bị phân tán thành
tại các vị trí khác nhau. Đồng thời Addon lại chia sẻ rất ít thông
tin cho các Addon khác.

Điều này làm nảy sinh một nhiệm vụ cần được xây dựng là tập hóa các thông
tin về quá trình kiểm thử, thông tin thu thập được thành một trang có sức
trực quan hóa cao để hiệu quả trong việc theo dõi.

Kiến trúc Event Bus của ZAP Core và các thông tin được cung cấp thêm trực
tiếp các Addon khác và các thành phần được thêm vào. Cần thu thập thông
tin lượng vừa đủ và chuẩn hóa các thông tin thu thập được thành một miền
cố định để có thể trực quan hóa. Như đã nói ở trên, Addons rất ít chia sẻ
thông tin ra ngoài nên lượng thông tin em cần thực hiện qua một số kĩ
thuật khác với luồng hoạt động thông thường của ZAP để có thể tổng hợp
được nhiều và đầy đủ hơn lượng thông tin \ref{fig:state}.

\begin{figure}[h!]
	\includegraphics[width=\linewidth]{./images/state.png}
	\caption{Thanh trạng thái pipeline đang chạy}
	\label{fig:state}
\end{figure}

\subsection{Quản lý Vấn đề và sinh báo cáo}
Theo luồng được cộng đồng ZAP định ra, thì cảnh báo (Alert) là đơn
vị được sử dụng để quản lý các lỗ hổng được sinh ra từ ZAP, tuy nhiên
điều này hơi không phù hợp do hai lý do. Một là lượng cảnh báo mà ZAP
sinh ra rất lớn và được sinh ra liên tục dẫn đến cần được xác thực lại
liên tục. Hai là mục tiêu cuối cùng là cần sinh ra được báo cáo bao gồm
thông tin chi tiết, cách tái hiện và gợi ý về cách xử lý cho phía
phát triển xử lý. Điều này dẫn để cần chuẩn hóa lại các miền để
phù hợp với mục đích cuối cùng là giúp nhà phát triển và vấn hành
xử lý được các lỗ hổng tồn tại.

Từ đó em đã mô hình hóa lại thành Vấn đề (Issue) để xử lý hai vấn đề ở trên:
chỉ sử dụng cảnh báo nhưng một nguồn và định thông tin tiêu chuẩn để
báo cáo xử lý và cho phép xuất ra một bản báo cáo dạng rich text để cung
cấp đủ thông tin về lỗ hổng \ref{fig:alert}.

Các cảnh báo có thể gửi tới các tab khai thác.

\begin{figure}[h!]
	\includegraphics[width=\linewidth]{./images/alert.png}
	\caption{Quản lý alert chung của các công cụ thích hợp}
	\label{fig:alert}
\end{figure}

Báo cáo sinh ra cho phép sử dụng Issue không phải trạng thái
UNKNOWN để tạo ra một File PDF chứa báo cáo với format định trước \ref{fig:report}.

\begin{figure}[h!]
	\includegraphics[width=\linewidth]{./images/report.png}
	\caption{Sinh báo cáo cho các lỗ hổng.}
	\label{fig:report}
\end{figure}

\subsection{Các công cụ khai thác và hậu khai thác}

Ngoài việc xác minh tự động trong đường ống, các công cụ thích
hợp còn có thể sử dụng như một công cụ khai thác tự động trong ZAP.
Việc thích hợp sử dụng chung các thành phần chính với công cụ xác minh
tự động.

Để thuận tiện cho việc khai thác thì có thể tạo ra nhiều nhãn (Tab)
khai thác khác nhau, mỗi nhãn này là các yêu cầu khai thác khác nhau.
Tuy nhiên, đặc điểm chung là các nhãn này sẽ không bao giờ được lưu lại
khi hết phiên để tránh lộ thông tin do các vấn đề về an toàn.
\subsubsection{SQLMap}
Với lỗ hổng này, người dùng có thể khai thác từ một cách tự động
bằng cách trích xuất các thông tin từ lỗ hổng \ref{fig:sqlmap_explot}.

\begin{figure}[h!]
	\includegraphics[width=\linewidth]{./images/sqlmap_explot.png}
	\caption{Khai thác lỗ hổng SQL Injection thông qua SQLMap}
	\label{fig:sqlmap_explot}
\end{figure}
Dữ liệu đầu vào có thể gửi trực tiếp từ các cảnh báo tới, các dữ
liệu như URL, Cookie, Data sẽ được gửi cùng sang để việc khai thác
trở nên ngắn gọn hơn hoặc tự chỉnh sửa để dễ dàng xác nhận hơn.
\subsubsection{Commix}

Dữ liệu đầu vào cũng có thể gửi từ cảnh báo sang tương tự SQLMap.

Các câu lệnh sẽ được gửi tới Commix và nhận về output tương tự một
trình pseudo-shell \ref{fig:commix_exploit}.

\begin{figure}[h!]
	\includegraphics[width=\linewidth]{./images/commix_exploit.png}
	\caption{Khai thác lỗ hổng OS Command Injection bằng Commix	}
	\label{fig:commix_exploit}
\end{figure}
\subsubsection{Tplmap}
Tương tự SQLMap và Commix, Tplmap  câu lệnh khởi tạo có thể được gửi
sang từ một cảnh báo.
Các câu lệnh trong Tplmap cũng được cài đặt để nhận một đầu vào và
một đầu ra nhưng một trình pseudo-shell \ref{fig:tplmap_exploit}.

\begin{figure}[h!]
	\includegraphics[width=\linewidth]{./images/tplmap_exploit.png}
	\caption{Khai thác lỗ hổng SSTI bằng Tplmap}
	\label{fig:tplmap_exploit}
\end{figure}

\subsubsection{Remote File Inclusion}

Việc khai thác Remote File Inclusion PHP được em triển khai bằng cách
thông qua một tập tin thực thi tự triển khai được công khai trên Internet.
Trong đây sẽ có một số tham số cho phép mình thực thi lệnh, các đầu ra
của lệnh sẽ được gửi về bằng cách phân tích phản hồi của yêu cầu \ref{fig:rfi_exploit}.

\begin{figure}[h!]
	\includegraphics[width=\linewidth]{./images/rfi_exploit.png}
	\caption{Khai thác lỗ hổng RFI bằng RFI Exploiter}
	\label{fig:rfi_exploit}
\end{figure}
\subsubsection{Local File Inclusion/Path Traversal}

Trình khai thác lỗ hổng Local File Inclusion/Path Traversal được
em triển khai bằng cách cho phép người dùng lựa chọn một số đường
dẫn và thay đổi các đường dẫn để lấy dữ liệu về, người dùng có thể
xem dữ liệu trả về thông qua việc quan sát phản hồi \ref{fig:lfi_exploit}.

\begin{figure}[h!]
	\includegraphics[width=\linewidth]{./images/lfi_exploit.png}
	\caption{Khai thác LFI/Path Traversal bằng công cụ LFI Exploiter}
	\label{fig:lfi_exploit}
\end{figure}
\subsection{Đóng gói các thành phần}

Ngoài việc cần có các Interface để giao tiếp với từng công cụ mã nguồn
mở khác nhau thì cũng cần đóng gói các công cụ mở rộng vào thành một
Addons hoàn chỉnh. Để giảm thiểu các cài đặt thêm để chạy được các công
cụ mở rộng thì em lựa chọn Docker để đóng gói.

Điều này có lợi ích sẽ giúp các công cụ được đóng gói toàn diện, chạy
được đa nền tảng tương tự ZAP. Docker cũng là một nền tảng ảo hóa được
sử dụng ngày càng phổ biến nên cho phép công cụ dễ dàng thích nghi hơn.

Sự lựa chọn tối ưu để đóng gói các thành phần mà vẫn đem lại hiệu xuất
tốt cho các công cụ mở rộng là sử dụng Docker.

Docker có cung cấp Interface để giao tiếp với Docker Daemon thông qua
REST-API. Java và Kotlin không được hỗ trợ chính thức các thư viện để
giao tiếp nhưng cộng đồng có các thư viện để giao tiếp tuy nhiên đã cũ
và có rất ít tài liệu hướng dẫn nên cần bọc lại cung cấp một API hoàn
chỉnh hơn cho các Component sử dụng \ref{fig:doc}.

Các API này bao gồm
\begin{itemize}
	\item Việc xây dựng các Docker Image chứa các công cụ ở ngoài.
	\item Tạo Container để phù hợp với các tác vụ.
	\item Giao tiếp với các Container.
	\item Xóa bỏ các Container không sử dụng.
\end{itemize}

Để sử dụng các Container hiệu quả, các các Container nên được đặt
có các Network Interface cùng với hệ điều hành để tránh việc kết nối
bị từ chối do Container sử dụng một mạng ảo riêng.

\begin{figure}[h!]
	\includegraphics[width=\linewidth]{./images/docker.png}
	\caption{Quản lý Docker của từng thành phần}
	\label{fig:docker}
\end{figure}

\end{document}