% !TeX root = ../main.tex
\documentclass[./../main.tex]{subfiles}

\begin{document}
\subsection{Zed Attack Proxy (ZAP)}
OWASP Zed Attack Proxy (thường được gọi tắt là  ZAP hay ZAProxy) là công cụ mã nguồn mở, công cụ cho phép các các kỹ sư an toàn thông tin tìm kiếm được lỗ hổng một cách tự động/bán tự động/thủ công thông qua cơ chế “middle-in-the-middle-proxy” giúp người dùng đứng giữa ứng dụng Web để thực hiện kiểm thử qua các plugin trong được xây dựng trong cộng đồng.

Điểm mạnh của ZAP là ứng dụng được viết bằng Java nên có thể triển khai trên nhiều nền tảng và có cộng đồng phát triển thường xuyên cập nhập các tri thức liên quan đến lỗ hổng hiện hành.

\subsection{ZAP Addons}
Điểm đặc biệt của ZAP là cơ chế mở rộng bằng ZAP Addons cho phép cộng đồng xây dựng được các phần mở rộng riêng cho mình trên ZAP mà không cần sửa các phần liên quan đến lõi của ZAP, đảm bảo sự độc lập tránh sự conflict giữa các ZAP Addons hoặc ZAP Addons với chính ZAP, đồng thời cũng đảm bảo được các phần được bảo trị độc lập một cách liên tục.
\subsubsection{Addon Spider và Addon Spider Ajax}
Một trong các Addons Core, được phát triển để Crawl và phát hiện được các thành phần đơn trong Website như các URL, action,...được gọi là các Site Node. Site Node cung cấp sẽ thông tin cho người dùng về thông tin Website đồng thời cũng cung cấp các thông tin về Website cho các Addon khác thông qua các API của ZAP.
\subsubsection{Addon Passive Scan và Active Scan}
Một Addons Core, được sử dụng để chạy các kiểm thử dạng được lập trình sẵn thành các Plugin đặc biệt gọi là Rule. Các Rule sẽ dựa theo thông tin của hệ thống cung cấp để chạy các Test Case và phân tích kết quả dựa trên các hành vi của Website để đưa ra các cảnh báo (Alert) cho người dùng.

Các Rule liên tục được cộng đồng duy trì và phát triển theo các lỗ hổng đã xuất hiện và hiện hành.
\subsection{SQL Map}
SQLMap là Một công cụ mã nguồn mở được sử dụng để tự động quá trình xác định, khai thác và hậu khai thác các lỗ hổng về SQL Injection.
SQLMap được cộng đồng phát triển với một engine có khả năng phát hiện các lỗ hổng ở một miền lỗ hổng cực kỳ rộng và cho rất nhiều hệ thống khác nhau. Đồng thời nó cũng có các công cụ khai thác dựa trên các lỗ hổng được phát hiện trên dump cơ sở dữ liệu hiệu quả.
\subsection{Commix}
Commix (viết tắt của command injection exploiter) là một công cụ mã nguồn mở được sử dụng để tự động xác định, khai thác và hậu khai thác các lỗ hổng về Command Injection. Một số lỗ hổng liên quan đến Remote Code Evaluation cũng có thể được phát hiện và khai thác bằng công cụ này.
Commix cũng được cộng đồng phát triển để có khả năng khai thác một miền rộng các lỗ hổng trên nhiều hệ điều hành khác nhau và có thể thực hiện hậu khai thác ở đây.
\subsection{Tplmap}
Tplmap là một công cụ mã nguồn mở được sử dụng để xác minh và khai thác các lỗ hổng liên quan đến Code Injection and Server-Side Template Injection.
Hơn 15 ngôn ngữ, template engine có thể bị khai thác với nhiều kỹ thuật khác nhau kể các các lỗ hổng dạng mù.

\end{document}